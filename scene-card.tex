\documentclass[letterpaper,10pt]{article}
\usepackage[margin=0.6in]{geometry}   % keep everything on a single page
\usepackage{array,tabularx,xcolor}
\renewcommand\arraystretch{1.2}

%%%%% Usage notes
%%%%%
%%%%%  Compile: pdflatex scene-card.tex (twice for good measure)
%%%%%  
%%%%%  Adjust area heights: change the [10em] arguments; e.g. [14em] for more writing space.
%%%%%  
%%%%%  Change widths: tweak 0.16\linewidth (label cells) and 0.09\linewidth
%%%%%  (blank cells) in the \newcolumntype directives.


% Fixed‑width label column, blank‑cell column, and a stretch column
\newcolumntype{L}{>{\raggedright\arraybackslash\bfseries}p{0.16\linewidth}}
\newcolumntype{B}{p{0.09\linewidth}}
\newcolumntype{X}{>{\raggedright\arraybackslash}X}

\begin{document}
\begin{center}
  {\Large\scshape Scene Card}
\end{center}

% ------------------------------------------------------------------
%  Top meta rows
\begin{tabularx}{\textwidth}{|L|B|L|B|L|B|L|B|}
\hline
MS/XP Earned  &  & Keywords            &  & Lure In &  & Lure Out &  \\ \hline
MS/XP Budget  &  & Motif \& Symbolism  &  & Lure In &  & Lure Out &  \\ \hline
\end{tabularx}

\vspace{1em}

% ------------------------------------------------------------------
%  Objective description block
\begin{tabularx}{\textwidth}{|L|X|}
\hline
Objective Description &
  \emph{The Challenge, The Interactions, The Dialogue, The Action, The Purpose}\\[10em] \hline
\end{tabularx}

\vspace{0.4em}

% ------------------------------------------------------------------
%  Sensory description block
\begin{tabularx}{\textwidth}{|L|X|}
\hline
Sensory Description &
  \emph{Lighting, Sounds, Visual, Smells, Tastes, Textures, Feel, Tension, Mood}\\[10em] \hline
\end{tabularx}

\vspace{0.4em}

% ------------------------------------------------------------------
%  Physical description block
\begin{tabularx}{\textwidth}{|L|X|}
\hline
Physical Description &
  \emph{Shape, Color, Materials, Length, Width, Height, Depth, Fixtures, Furnishings}\\[10em] \hline
\end{tabularx}

\vspace{1em}

% ------------------------------------------------------------------
%  Action / detail grid
\begin{tabularx}{\textwidth}{|L|B|L|B|L|B|L|B|}
\hline
Plan              & & Combat?       & & Setback & & Hints  & \\ \hline
Ploy              & & Resource Use  & & Objects & & Clues  & \\ \hline
Goal | Objective  & & Minions       & & Lore    & & Hoard  & \\ \hline
\end{tabularx}

\end{document}
